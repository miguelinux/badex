% ex: ts=2 sw=2 sts=2 et filetype=tex
% SPDX-License-Identifier: CC-BY-SA-4.0

%\textbf{Costo del transporte público}. 
\question En una localidad, el costo
          de transporte público es proporcional a la cantidad de viajes que
          realizas. Si un solo viaje cuenta \$ 8.50 responde:

% Usa letras minúsculas en lugar de números
\renewcommand{\labelenumi}{\alph{enumi})}
\begin{enumerate}
  \item ¿Cuánto pagarías por 6 vieajes? \fillin[51]
  \item ¿Y si tomaras 10 vieajes? \fillin[85]
  \item Expresa esta relación en forma de ecuación (considera que \textbf{x}
        representa el número de viajes y \textbf{y} el costo total a
        pagar). \fillin[$y = 8.5x$]
  \item completa la tabla y grafica la relación que hay entre el número
        de viajes y el costo.

\begin{multicols}{2}
\begin{center}
  \begin{tabular}{|c|c|}
     \hline
     \rowcolor[HTML]{C0C0C0}
         Viajes & Costo \\
     \hline 0 &  \\
     \hline 1 &  \\
     \hline 2 &  \\
     \hline 3 &  \\
     \hline 4 &  \\
     \hline 5 &  \\
     \hline
  \end{tabular}
\end{center}

\begin{tikzpicture}
  \coordinate (O) at (0,0);
  \draw[->] (-0.3,0) -- (3,0) coordinate[label = {below:$x$}] (xmax);
  \draw[->] (0,-0.3) -- (0,3) coordinate[label = {right:$y$}] (ymax);

\end{tikzpicture}
\end{multicols}


  \item ¿En qué punto la recta se intersecta con el eje \textbf{Y}?
        \fillin[(0,0)]
\end{enumerate}
