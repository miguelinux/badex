% ex: ts=2 sw=2 sts=2 et filetype=tex
% SPDX-License-Identifier: CC-BY-SA-4.0

%\textbf{Ahorro de monedas}. 
\question Con el fin de hacer un ahorro, Cecilia
          decide guardar diariamente monedas de \$ 10  en su alcancía. Si
          comienza con 5 monedas en su alcancía y las monedas que ahorra las
          determina la expreción $y = 2x + 5$, calcula cuántas monedas
          colocará en su alcancía en los días 1, 2,3, 4 y 5. Grafica la
          expreción.

\textbf{Tabla de valores}

\begin{multicols}{2}
\begin{center}
  \begin{tabular}{|c|c|}
     \hline
     \rowcolor[HTML]{C0C0C0}
         x & $y = 2x + 5$ \\
     \hline 1 &  \\
     \hline 2 &  \\
     \hline 3 &  \\
     \hline 4 &  \\
     \hline 5 &  \\
     \hline
  \end{tabular}
\end{center}

\begin{tikzpicture}
  \coordinate (O) at (0,0);
  \draw[->] (-0.3,0) -- (3,0) coordinate[label = {below:$x$}] (xmax);
  \draw[->] (0,-0.3) -- (0,3) coordinate[label = {right:$y$}] (ymax);
\end{tikzpicture}
\end{multicols}
