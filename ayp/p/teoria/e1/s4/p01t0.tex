% ex: ts=2 sw=2 sts=2 et filetype=tex
% SPDX-License-Identifier: CC-BY-SA-4.0

\question Escribir un programa que calcule la distancia entre dos puntos en
          un plano cartesiano utilizando el teorema de Pitagoras. La
          distancia entre dos puntos $P_1$ y $P_2$ del plano se denotará por
          $d(P_1,P_2)$ donde $P_1=(x_1,y_1)$ y $P_2=(x_2,y_2)$. La fórmula es:

\begin{tikzpicture}
  \coordinate (O) at (0,0);
  \draw[->] (-0.3,0) -- (6,0) coordinate[label = {below:$x$}] (xmax);
  \draw[->] (0,-0.3) -- (0,4) coordinate[label = {right:$y$}] (ymax);
  \draw[dashed] (1,1) -- (4,1) -- (4,3);
  \draw[thick]  (1,1) node[anchor=north west] {$P_1=(x_1,y_1)$ }-- (4,3) node[anchor=south] {$P_2=(x_2,y_2)$};
  \draw[densely dotted]  (1,0) node[anchor=north] {$x_1$} -- (1,1) ;
  \draw[densely dotted]  (4,0) node[anchor=north] {$x_2$} -- (4,1);
  \draw[densely dotted]  (0,1) node[anchor=east] {$y_1$} -- (1,1);
  \draw[densely dotted]  (0,3) node[anchor=east] {$y_2$} -- (4,3);
  \node[circle, fill,inner sep=1.5pt] at (1,1) {};
  \node[circle, fill,inner sep=1.5pt] at (4,3) {};
  \node[anchor=west] at (5,2) {$d(P_1, P_2) = \sqrt{(x_2 - x_1)^2 + (y_2 - y_1)^2}$};
\end{tikzpicture}

\textbf{Entrada}

Cuatro números enteros que representan dos puntos $P_1$ y $P_2$ en el plano
cartesiano. El primer número es $x_1$, el segundo $y_1$, el tercero $x_2$ y
el cuarto $y_2$.

\textbf{Salida}

El valor real que represanta la distancia entre los dos puntos. el mensaje a
imprimir es: $d(P_1,P_2)=$<valor\_numérico>.

\begin{center}
  \begin{tabular}{|c|c|c|}
     \hline
     \rowcolor[HTML]{C0C0C0}
             & Ejemplo 1 & Ejemplo 2 \\
     \hline
     Entrada & 1         &  3 \\
             & 1         &  4 \\
             & 2         &  2 \\
             & 3         &  1 \\
     \hline
     Salida  & d(P1,P2) = 2.2361 & d(P1,P2) = 3.1623 \\
     \hline
  \end{tabular}
\end{center}
          
