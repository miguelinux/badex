% ex: ts=2 sw=2 sts=2 et filetype=tex
% SPDX-License-Identifier: CC-BY-SA-4.0

\question Calcular la ecuación de segundo grado $ax^2 + bx + c$, con sus dos
          soluciones o raíces según aplique. Utilizando la fórmula general:

\begin{displaymath}
  x_{1,2} = \frac{-b \pm \sqrt{b^2 - 4ac}}{2a}
\end{displaymath}

Considerando que para que la ecuación tenga solución es preciso que el
\textbf{discriminante} sea mayor o igual que cero. El discriminante de una
ecuación de segundo grado es: $d = b^2 - 4ac$, por lo tanto si $d=0$ se
obtienen dos raíces iguales $x_1 = x_2$ y se dice que la ecuación tiene una
solución, si $d >0$ se calculan las dos raices $x1$ y $x2$, y se dice que la
ecuación tiene dos soluciones, y si $d<0$ entonces la ecuación no tiene
solución.

\textbf{Entrada}

Valor del coeficiente $a$. \\
Valor del coeficiente $b$. \\
Valor del coeficiente $c$. \\

\textbf{Salida}

El mensaje indicando si la ecuación tiene \textbf{cero}, \textbf{una} o
\textbf{dos} soluciones.

\begin{center}
  \begin{tabular}{|c|c|c|c|}
     \hline
     \rowcolor[HTML]{C0C0C0}
             & Ejemplo 1 & Ejemplo 2 & Ejemplo 3 \\
     \hline
     Entrada & 1   &  1  &  1 \\
             & 0   &  -4  & 1 \\
             & 1   &  4  & -2 \\
     \hline
     Salida  & No tiene solución & Una solución & Dos soluciones \\
     \hline
  \end{tabular}
\end{center}
