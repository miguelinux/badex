% ex: ts=2 sw=2 sts=2 et filetype=tex
% SPDX-License-Identifier: CC-BY-SA-4.0

\question Estas a cargo del desarrollo de una aplicación que sirve para
          encontrar pokemón. Tienes una lista en orden (“pokemon.txt”)
          de todos los pokemón, donde tienes en una columna el número de
          pokemón y a la derecha, separado por un espacio, el nombre de
          dicho pokemón.

Existen dos casos de uso:
\begin{enumerate}
  \item Que el usuario te dé un número y regresarle el nombre del pokemón
  \item Que el usuario te dé el nombre del pokemón y regresarle un número de
        pokemón
\end{enumerate}

Por supuesto, debes tener mensajes de error en caso de que no se encuentre
lo que el usuario busca.

\textbf{Entrada}

\begin{itemize}
  \item Archivo txt
  \item Un número o un nombre
\end{itemize}

\textbf{Salida}

\begin{itemize}
  \item Un número o un nombre
\end{itemize}
