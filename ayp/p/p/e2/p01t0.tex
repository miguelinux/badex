% ex: ts=2 sw=2 sts=2 et filetype=tex
% SPDX-License-Identifier: CC-BY-SA-4.0

\question El máximo común divisor de dos números (enteros) es el mayor
          número que los divide sin residuo. Cuando dos número sólo
          comparten el 1 como divisor, se dice que son "\textbf{coprimos}".
          Escribir un programa que dado dos números enteros, obtenga su
          máximo común divisor, y determine si son "\emph{coprimos}".

\textbf{Entrada}

Dos números enteros mayores a cero.

\textbf{Salida}

Su máximo común divisor. Y en caso de ser 1, imprimir: "Los números
\emph{<numero1>} y \emph{<numero2>} son coprimos".

\begin{center}
  \begin{tabular}{|c|l|c|}
     \hline
     \rowcolor[HTML]{C0C0C0}
             & Ejemplo 1 & Ejemplo 2 \\
     \hline
     Entrada & 4         &  21 \\
             & 15        &  42 \\
     \hline
     Salida  & 1 & 21 \\
             & Los números 4 y 15 son coprimos & \\
     \hline
  \end{tabular}
\end{center}

