% ex: ts=2 sw=2 sts=2 et filetype=tex
% SPDX-License-Identifier: CC-BY-SA-4.0

\question Suma de matrices. La suma de dos matrices (A + B) requiere que
          ambas matrices tengan la misma dimensión. La matriz resultante (C)
          también será de la misma dimensión. Ejemplo:

\begin{displaymath}
  A = 
  \begin{pmatrix}
    2 & 0 & 1 \\
    3 & 0 & 0 \\
    5 & 1 & 1
  \end{pmatrix},
  B=
  \begin{pmatrix}
    1 & 0 & 1 \\
    1 & 2 & 1 \\
    1 & 1 & 0
  \end{pmatrix},
  A + B =
  \begin{pmatrix}
    2+1 & 0+0 & 1+1 \\
    3+1 & 0+2 & 0+1 \\
    5+1 & 1+1 & 1+0
  \end{pmatrix}=
  \begin{pmatrix}
    3 & 0 & 2 \\
    4 & 2 & 1 \\
    6 & 2 & 1
  \end{pmatrix}
\end{displaymath}

\textbf{Entrada}

Dos matrices cuadradas A y B de orden n. Los datos de las matrices a sumar A
+ B serán asignados de forma aleatoria en un rango de 0 a 10.

\textbf{Salida}

\begin{enumerate}
  \item Matriz resultante C.
  \item Número mayor de todos los elementos de la matriz resultante.
  \item Número menor de todos los elementos de la matriz resultante.
  \item Elementos de la Diagonal Principal de la matriz resultante.
  \item Elementos de la Diagonal Secundaria de la matriz resultante.
\end{enumerate}
