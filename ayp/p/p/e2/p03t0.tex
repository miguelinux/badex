% ex: ts=2 sw=2 sts=2 et filetype=tex
% SPDX-License-Identifier: CC-BY-SA-4.0

\question  Escribir un programa que simule el comportamiento de una pila.
           Una pila es una lista ordenada de elementos, donde el último
           elmento que entra es el primer elemento que sale (contrario a una
           cola). Realizar las siguientes operaciones en el orden descrito:

\begin{enumerate}
  \item Apilar elementos de tipo numérico a la pila. Tantos como decida el
        usuario.
  \item Desapilar un elmento.
  \item Apilar un nuevo elmento.
  \item Imprimir la pila.
  \item Imprimir el tamaño de la pila.
\end{enumerate}

\begin{center}
  \begin{tabular}{l|c|cr}
    Último elmento apilado  & 44 & 5 & \\
       & 12 & 4 & \\
       & 34 & 3 & \\
       & 51 & 2 & \\
       &  9 & 1 & \\
       & 28 & 0 & Primer elmento apilado\\
    \hline
       & Pila & &
  \end{tabular}
\end{center}
