% ex: ts=2 sw=2 sts=2 et filetype=tex
% SPDX-License-Identifier: CC-BY-SA-4.0

\documentclass[10pt,addpoints]{exam}

\usepackage[utf8]{inputenc}
\usepackage[T1]{fontenc}
\usepackage[letterpaper,top=50mm]{geometry}
\usepackage{graphicx}
\usepackage{xcolor}
%\usepackage[spanish]{babel}

\definecolor{itesoprofundo}{RGB}{000, 066, 112}
\definecolor{itesomedio}{RGB}{112, 144, 183}
\definecolor{itesoclaro}{RGB}{151, 227, 255}

\pagestyle{headandfoot}
%\headrule
\lhead{\includegraphics[scale=0.5]{img/logo.png}}
\rhead{\color{itesoprofundo}\textbf{ALGORITMOS Y PROGRAMACIÓN}\\Departamento
       de Electrónica, Sistemas e Informática\\}
\footer{}{Página \thepage\ de \numpages}{}

\pointpoints{punto}{puntos}
\renewcommand{\solutiontitle}{\textbf{Solución: }}

%\printanswers

\begin{document}
\begin{center}
  \sffamily\textbf{EXAMEN I}
\end{center}
\begin{flushright}
8 de octubre de 2023
\end{flushright}
Nombre:~\makebox[4in]{\hrulefill}
Puntos máximos: 100

\begin{questions}

\begin{EnvFullwidth}
  \sffamily\textbf{Sección I.- Completar los siguientes enunciados con las
  palabras situadas en la parte inferior de la sección (valor 15 puntos).}
\end{EnvFullwidth}

% ex: ts=2 sw=2 sts=2 et filetype=tex
% SPDX-License-Identifier: CC-BY-SA-4.0

\question Conjunto de pasos, acciones o instrucciones necesarios para lograr 
un resultado  o resolver  un problema.

  \begin{oneparchoices}
    \CorrectChoice Algoritmo
    \choice Problema
    \choice Solución
    \choice Reseta de cocina
  \end{oneparchoices}

% ex: ts=2 sw=2 sts=2 et filetype=tex
% SPDX-License-Identifier: CC-BY-SA-4.0

\question Es cada uno de los elementos que componen la población:

  \begin{oneparchoices}
    \CorrectChoice Individuo
    \choice Dato
    \choice Entrevista
  \end{oneparchoices}

% ex: ts=2 sw=2 sts=2 et filetype=tex
% SPDX-License-Identifier: CC-BY-SA-4.0

\question Los \fillin\ se caracterizan porque las partículas que
los componen están muy cercanas entre sí, y en posiciones más o menos fijas

  \begin{oneparchoices}
    \choice Líquidos
    \choice Gaseosos
    \CorrectChoice Sólidos
    \choice Todos
  \end{oneparchoices}
  \answerline[C]

% ex: ts=2 sw=2 sts=2 et filetype=tex
% SPDX-License-Identifier: CC-BY-SA-4.0

\question Como se llaman los lados que forman el ángulo recto en un
          triángulo rectángulo.

  \begin{oneparchoices}
    \choice Lado de 90 grados
    \choice Lado recto
    \choice Hipotenusa
    \CorrectChoice Cateto
  \end{oneparchoices}
  \answerline[D]

% ex: ts=2 sw=2 sts=2 et filetype=tex
% SPDX-License-Identifier: CC-BY-SA-4.0

\question Es todo aquello que puede ser medido, como el tiempo, la longitud,
  la masa, el área, el volumen, la densidad, la fuerza, etc. y se representa
  con un nú mero y una unidad.

  \begin{oneparchoices}
    \choice Metro
    \choice Litro
    \CorrectChoice Magnitud
    \choice Regla
  \end{oneparchoices}
  \answerline[C]

% ex: ts=2 sw=2 sts=2 et filetype=tex
% SPDX-License-Identifier: CC-BY-SA-4.0

\question Encuentra $f(2)$ para la función: $f(x) = 3x + 2$:

  \begin{oneparchoices}
    \choice  $f(2) = 6$
    \CorrectChoice $f(2) = 8$
    \choice $f(2) = 10$
  \end{oneparchoices}

% ex: ts=2 sw=2 sts=2 et filetype=tex
% SPDX-License-Identifier: CC-BY-SA-4.0

\question \[ \frac{8}{4} + \frac{9}{4} = \frac{\hspace{2cm}}{\hspace{2cm}} 
             \hspace{2cm}
             \frac{5}{2} + \frac{4}{3} = \frac{\hspace{2cm}}{\hspace{2cm}}
          \]


% ex: ts=2 sw=2 sts=2 et filetype=tex
% SPDX-License-Identifier: CC-BY-SA-4.0

\question \[ \frac{7}{5} - \frac{5}{5} = \frac{\hspace{2cm}}{\hspace{2cm}}
             \hspace{2cm}
             \frac{2}{2} - \frac{1}{4} = \frac{\hspace{2cm}}{\hspace{2cm}}
          \]

% ex: ts=2 sw=2 sts=2 et filetype=tex
% SPDX-License-Identifier: CC-BY-SA-4.0

\question $8xy(9x^3y - 6x^2y - 3xy^3) = $

  \begin{oneparchoices}
    \CorrectChoice $72x^4y^2 - 48x^3y^2 - 24x^2y^4$
    \choice $72x^4y^2 + 48x^3y^2 + 24x^2y^4$
    \choice $17x^4y^2 - 14x^3y^2 - 12x^2y^4$
    \choice $17x^4y^2 + 14x^3y^2 + 12x^2y^4$
  \end{oneparchoices}

% ex: ts=2 sw=2 sts=2 et filetype=tex
% SPDX-License-Identifier: CC-BY-SA-4.0

\question Es solo una pequeña parte de la población:

  \begin{oneparchoices}
    \CorrectChoice Muestra
    \choice Individuo
    \choice Dato
  \end{oneparchoices}

% ex: ts=2 sw=2 sts=2 et filetype=tex
% SPDX-License-Identifier: CC-BY-SA-4.0

\question Conjunto de pasos, acciones o instrucciones necesarios para lograr 
un resultado  o resolver  un problema.

  \begin{oneparchoices}
    \CorrectChoice Algoritmo
    \choice Problema
    \choice Solución
    \choice Reseta de cocina
  \end{oneparchoices}

% ex: ts=2 sw=2 sts=2 et filetype=tex
% SPDX-License-Identifier: CC-BY-SA-4.0

\question Es cada uno de los elementos que componen la población:

  \begin{oneparchoices}
    \CorrectChoice Individuo
    \choice Dato
    \choice Entrevista
  \end{oneparchoices}

% ex: ts=2 sw=2 sts=2 et filetype=tex
% SPDX-License-Identifier: CC-BY-SA-4.0

\question Los \fillin\ se caracterizan porque las partículas que
los componen están muy cercanas entre sí, y en posiciones más o menos fijas

  \begin{oneparchoices}
    \choice Líquidos
    \choice Gaseosos
    \CorrectChoice Sólidos
    \choice Todos
  \end{oneparchoices}
  \answerline[C]

% ex: ts=2 sw=2 sts=2 et filetype=tex
% SPDX-License-Identifier: CC-BY-SA-4.0

\question Como se llaman los lados que forman el ángulo recto en un
          triángulo rectángulo.

  \begin{oneparchoices}
    \choice Lado de 90 grados
    \choice Lado recto
    \choice Hipotenusa
    \CorrectChoice Cateto
  \end{oneparchoices}
  \answerline[D]

% ex: ts=2 sw=2 sts=2 et filetype=tex
% SPDX-License-Identifier: CC-BY-SA-4.0

\question Es todo aquello que puede ser medido, como el tiempo, la longitud,
  la masa, el área, el volumen, la densidad, la fuerza, etc. y se representa
  con un nú mero y una unidad.

  \begin{oneparchoices}
    \choice Metro
    \choice Litro
    \CorrectChoice Magnitud
    \choice Regla
  \end{oneparchoices}
  \answerline[C]

% ex: ts=2 sw=2 sts=2 et filetype=tex
% SPDX-License-Identifier: CC-BY-SA-4.0

\question Encuentra $f(2)$ para la función: $f(x) = 3x + 2$:

  \begin{oneparchoices}
    \choice  $f(2) = 6$
    \CorrectChoice $f(2) = 8$
    \choice $f(2) = 10$
  \end{oneparchoices}

% ex: ts=2 sw=2 sts=2 et filetype=tex
% SPDX-License-Identifier: CC-BY-SA-4.0

\question \[ \frac{8}{4} + \frac{9}{4} = \frac{\hspace{2cm}}{\hspace{2cm}} 
             \hspace{2cm}
             \frac{5}{2} + \frac{4}{3} = \frac{\hspace{2cm}}{\hspace{2cm}}
          \]


% ex: ts=2 sw=2 sts=2 et filetype=tex
% SPDX-License-Identifier: CC-BY-SA-4.0

\question \[ \frac{7}{5} - \frac{5}{5} = \frac{\hspace{2cm}}{\hspace{2cm}}
             \hspace{2cm}
             \frac{2}{2} - \frac{1}{4} = \frac{\hspace{2cm}}{\hspace{2cm}}
          \]

% ex: ts=2 sw=2 sts=2 et filetype=tex
% SPDX-License-Identifier: CC-BY-SA-4.0

\question $8xy(9x^3y - 6x^2y - 3xy^3) = $

  \begin{oneparchoices}
    \CorrectChoice $72x^4y^2 - 48x^3y^2 - 24x^2y^4$
    \choice $72x^4y^2 + 48x^3y^2 + 24x^2y^4$
    \choice $17x^4y^2 - 14x^3y^2 - 12x^2y^4$
    \choice $17x^4y^2 + 14x^3y^2 + 12x^2y^4$
  \end{oneparchoices}

% ex: ts=2 sw=2 sts=2 et filetype=tex
% SPDX-License-Identifier: CC-BY-SA-4.0

\question Es solo una pequeña parte de la población:

  \begin{oneparchoices}
    \CorrectChoice Muestra
    \choice Individuo
    \choice Dato
  \end{oneparchoices}


\end{questions}

\end{document}
