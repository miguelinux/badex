% ex: ts=2 sw=2 sts=2 et filetype=tex
% SPDX-License-Identifier: CC-BY-SA-4.0

\documentclass[10pt,addpoints]{exam}

\usepackage[utf8]{inputenc}
\usepackage[T1]{fontenc}
\usepackage[letterpaper,top=50mm]{geometry}
\usepackage{graphicx}
\usepackage[table]{xcolor}
%\usepackage[spanish]{babel}
\usepackage{multicol}
\usepackage{amsmath}
\usepackage{listings}

\usepackage{tikz}
\newcommand{\pythagwidth}{3cm}
\newcommand{\pythagheight}{2cm}

\definecolor{itesoprofundo}{RGB}{000, 066, 112}
\definecolor{itesomedio}{RGB}{112, 144, 183}
\definecolor{itesoclaro}{RGB}{151, 227, 255}

\pagestyle{headandfoot}
%\headrule
\lhead{\includegraphics[scale=0.5]{img/logo.png}}
\rhead{\color{itesoprofundo}\textbf{DISEÑO DE ALGORITMOS}\\Departamento
       de Electrónica, Sistemas e Informática\\}
\lfoot{Nombre:~\makebox[4.5in]{\hrulefill} }
\cfoot{}
\rfoot{Página \thepage\ de \numpages}

\pointpoints{punto}{puntos}
\renewcommand{\solutiontitle}{\textbf{Solución: }}
\renewcommand{\theenumi}{\alph{enumi})}

\newcommand{\tf}[1][{}]{%
  \fillin[#1][0.25in]%
}

%\printanswers

\begin{document}
\begin{center}
  \sffamily\textbf{EXAMEN III}
\end{center}
\begin{flushright}
6 de mayo de 2024
\end{flushright}

\begin{questions}
\begin{EnvFullwidth}
  \sffamily\textbf{Sección I.- Analizar las instrucciones y determinar
  cual es el valor de salida que se obtiene al ejecutar el
  siguiente códigos (valor 30 puntos). Incluir las corridas de
  escritorio de las variables implicadas.
  }
\end{EnvFullwidth}

% ex: ts=2 sw=2 sts=2 et filetype=tex
% SPDX-License-Identifier: CC-BY-SA-4.0

\question Función que divide una cadena en una lista especificando un
          separador:
          \fillin[split()]


\end{questions}

\begin{questions}
\begin{EnvFullwidth}
  \sffamily\textbf{Sección II.- Resolver los siguientes problemas utilizando
  instrucciones del lenguaje Python. Valor 70 puntos (35 puntos c/u).}
\end{EnvFullwidth}

% ex: ts=2 sw=2 sts=2 et filetype=tex
% SPDX-License-Identifier: CC-BY-SA-4.0

\question \textbf{join()}: Función que retorna una cadena formada por los
          elementos de una lista separados por un delimitador.

% ex: ts=2 sw=2 sts=2 et filetype=tex
% SPDX-License-Identifier: CC-BY-SA-4.0

\question  


\end{questions}

\end{document}
