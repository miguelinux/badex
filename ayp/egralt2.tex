% ex: ts=2 sw=2 sts=2 et filetype=tex
% SPDX-License-Identifier: CC-BY-SA-4.0

\documentclass[10pt,addpoints]{exam}

\usepackage[utf8]{inputenc}
\usepackage[T1]{fontenc}
\usepackage[letterpaper]{geometry}
\usepackage{graphicx}
\usepackage[table]{xcolor}
%\usepackage[spanish]{babel}
\usepackage{multicol}
\usepackage{amsmath}
\usepackage{listings}

\usepackage{tikz}
\newcommand{\pythagwidth}{3cm}
\newcommand{\pythagheight}{2cm}

\pagestyle{headandfoot}
%\headrule
\lhead{\includegraphics[scale=0.10]{img/tsj.png}}
\rhead{\textbf{Diseño y Análisis de Algoritmos}}
\lfoot{Nombre:~\makebox[4.5in]{\hrulefill} }
\cfoot{}
\rfoot{Página \thepage\ de \numpages}

\pointpoints{punto}{puntos}
\renewcommand{\solutiontitle}{\textbf{Solución: }}
\renewcommand{\theenumi}{\alph{enumi})}

%\printanswers

\begin{document}
\begin{center}
  \sffamily\textbf{EXAMEN II}
\end{center}
\begin{flushright}
9 de abril de 2024
\end{flushright}

\begin{questions}
\begin{EnvFullwidth}
  \sffamily\textbf{Sección I.- Completar los siguientes enunciados.
  Valor: 60 puntos (5 punto c/u).}
\end{EnvFullwidth}

% ex: ts=2 sw=2 sts=2 et filetype=tex
% SPDX-License-Identifier: CC-BY-SA-4.0

\question Función que devuelve el tipo de dato de una variable:
          \fillin[type()]

% ex: ts=2 sw=2 sts=2 et filetype=tex
% SPDX-License-Identifier: CC-BY-SA-4.0

\question Manejo de espacios al principio de la liena para delimitar
          bloques de código:
          \fillin[Sangría]


% ex: ts=2 sw=2 sts=2 et filetype=tex
% SPDX-License-Identifier: CC-BY-SA-4.0

\question Estas a cargo del desarrollo de una aplicación que sirve para
          encontrar pokemón. Tienes una lista en orden (“pokemon.txt”)
          de todos los pokemón, donde tienes en una columna el número de
          pokemón y a la derecha, separado por un espacio, el nombre de
          dicho pokemón.

Existen dos casos de uso:
\begin{enumerate}
  \item Que el usuario te dé un número y regresarle el nombre del pokemón
  \item Que el usuario te dé el nombre del pokemón y regresarle un número de
        pokemón
\end{enumerate}

Por supuesto, debes tener mensajes de error en caso de que no se encuentre
lo que el usuario busca.

\textbf{Entrada}

\begin{itemize}
  \item Archivo txt
  \item Un número o un nombre
\end{itemize}

\textbf{Salida}

\begin{itemize}
  \item Un número o un nombre
\end{itemize}

% ex: ts=2 sw=2 sts=2 et filetype=tex
% SPDX-License-Identifier: CC-BY-SA-4.0

\question 
\begin{itemize}
  \item Abrir el archivo \textbf{nuevo.txt} como sólo lectura
  \item Asignar el nombre \underline{archivo}
  \item Localizar el puntero en la posición 12 del archivo
  \item Leer todas las líneas de texto a partir de dicha posición
  \item Desplegar cada línea
  \item Cerrar el archivo
\end{itemize}

% ex: ts=2 sw=2 sts=2 et filetype=tex
% SPDX-License-Identifier: CC-BY-SA-4.0

\question Qué representa: "[12, 8, 7, 9]":
          \fillin[cadena]

% ex: ts=2 sw=2 sts=2 et filetype=tex
% SPDX-License-Identifier: CC-BY-SA-4.0

\question Dibujar un cuadrado con ceros y unos, donde los ceros y unos
          no esten juntos y el tamaño del cuadrado será de $n \times n$,
          donde $n$ será dado por el usuario.

\textbf{Entrada}

Un números entero mayor a cero.

\textbf{Salida}

Un cuadrado generado con ceros y unos de tamaño del entero ingresado.

\begin{center}
  \begin{tabular}{|c|l|c|}
     \hline
     \rowcolor[HTML]{C0C0C0}
             & Ejemplo 1 & Ejemplo 2 \\
     \hline
     Entrada & 2         &  3 \\
     \hline
     Salida  & 01 & 010 \\
             & 10 & 101 \\
             &    & 010 \\
     \hline
  \end{tabular}
\end{center}


% ex: ts=2 sw=2 sts=2 et filetype=tex
% SPDX-License-Identifier: CC-BY-SA-4.0

\question \tf[V] El modo "r" abre un archivo de solo lectura y envía mensaje
          de error si el archivo no existe.


% ex: ts=2 sw=2 sts=2 et filetype=tex
% SPDX-License-Identifier: CC-BY-SA-4.0

\question Estas a cargo del desarrollo de una aplicación que sirve para
          encontrar pokemón. Tienes una lista en orden (“pokemon.txt”)
          de todos los pokemón, donde tienes en una columna el número de
          pokemón y a la derecha, separado por un espacio, el nombre de
          dicho pokemón.

Existen dos casos de uso:
\begin{enumerate}
  \item Que el usuario te dé un número y regresarle el nombre del pokemón
  \item Que el usuario te dé el nombre del pokemón y regresarle un número de
        pokemón
\end{enumerate}

Por supuesto, debes tener mensajes de error en caso de que no se encuentre
lo que el usuario busca.

\textbf{Entrada}

\begin{itemize}
  \item Archivo txt
  \item Un número o un nombre
\end{itemize}

\textbf{Salida}

\begin{itemize}
  \item Un número o un nombre
\end{itemize}

% ex: ts=2 sw=2 sts=2 et filetype=tex
% SPDX-License-Identifier: CC-BY-SA-4.0

\question 
\begin{itemize}
  \item Abrir el archivo \textbf{nuevo.txt} como sólo lectura
  \item Asignar el nombre \underline{archivo}
  \item Localizar el puntero en la posición 12 del archivo
  \item Leer todas las líneas de texto a partir de dicha posición
  \item Desplegar cada línea
  \item Cerrar el archivo
\end{itemize}

% ex: ts=2 sw=2 sts=2 et filetype=tex
% SPDX-License-Identifier: CC-BY-SA-4.0

\question Qué representa: "[12, 8, 7, 9]":
          \fillin[cadena]

% ex: ts=2 sw=2 sts=2 et filetype=tex
% SPDX-License-Identifier: CC-BY-SA-4.0

\question Dibujar un cuadrado con ceros y unos, donde los ceros y unos
          no esten juntos y el tamaño del cuadrado será de $n \times n$,
          donde $n$ será dado por el usuario.

\textbf{Entrada}

Un números entero mayor a cero.

\textbf{Salida}

Un cuadrado generado con ceros y unos de tamaño del entero ingresado.

\begin{center}
  \begin{tabular}{|c|l|c|}
     \hline
     \rowcolor[HTML]{C0C0C0}
             & Ejemplo 1 & Ejemplo 2 \\
     \hline
     Entrada & 2         &  3 \\
     \hline
     Salida  & 01 & 010 \\
             & 10 & 101 \\
             &    & 010 \\
     \hline
  \end{tabular}
\end{center}


% ex: ts=2 sw=2 sts=2 et filetype=tex
% SPDX-License-Identifier: CC-BY-SA-4.0

\question \tf[V] El modo "r" abre un archivo de solo lectura y envía mensaje
          de error si el archivo no existe.

\end{questions}

\begin{questions}
\begin{EnvFullwidth}
  \sffamily\textbf{Sección II.- Resolver el siguiente problemas utilizando
  instrucciones del lenguaje Python. Valor 40 puntos.}
\end{EnvFullwidth}

% ex: ts=2 sw=2 sts=2 et filetype=tex
% SPDX-License-Identifier: CC-BY-SA-4.0

\question Estas a cargo del desarrollo de una aplicación que sirve para
          encontrar pokemón. Tienes una lista en orden (“pokemon.txt”)
          de todos los pokemón, donde tienes en una columna el número de
          pokemón y a la derecha, separado por un espacio, el nombre de
          dicho pokemón.

Existen dos casos de uso:
\begin{enumerate}
  \item Que el usuario te dé un número y regresarle el nombre del pokemón
  \item Que el usuario te dé el nombre del pokemón y regresarle un número de
        pokemón
\end{enumerate}

Por supuesto, debes tener mensajes de error en caso de que no se encuentre
lo que el usuario busca.

\textbf{Entrada}

\begin{itemize}
  \item Archivo txt
  \item Un número o un nombre
\end{itemize}

\textbf{Salida}

\begin{itemize}
  \item Un número o un nombre
\end{itemize}


\end{questions}

\end{document}
