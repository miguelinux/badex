% ex: ts=2 sw=2 sts=2 et filetype=tex
% SPDX-License-Identifier: CC-BY-SA-4.0

\documentclass[10pt,addpoints]{exam}

\usepackage[utf8]{inputenc}
\usepackage[T1]{fontenc}
\usepackage[letterpaper]{geometry}
\usepackage{graphicx}
\usepackage[table]{xcolor}
%\usepackage[spanish]{babel}
\usepackage{multicol}
\usepackage{amsmath}
\usepackage{listings}

\usepackage{tikz}
\newcommand{\pythagwidth}{3cm}
\newcommand{\pythagheight}{2cm}

\definecolor{itesoprofundo}{RGB}{000, 066, 112}
\definecolor{itesomedio}{RGB}{112, 144, 183}
\definecolor{itesoclaro}{RGB}{151, 227, 255}

\pagestyle{headandfoot}
%\headrule
\lhead{\includegraphics[scale=0.5]{img/logo.png}}
\rhead{\color{itesoprofundo}\textbf{ALGORITMOS Y PROGRAMACIÓN}\\Departamento
       de Electrónica, Sistemas e Informática\\}
\lfoot{Nombre:~\makebox[4.5in]{\hrulefill} }
\cfoot{}
\rfoot{Página \thepage\ de \numpages}

\pointpoints{punto}{puntos}
\renewcommand{\solutiontitle}{\textbf{Solución: }}
\renewcommand{\theenumi}{\alph{enumi})}

\newcommand{\tf}[1][{}]{%
  \fillin[#1][0.25in]%
}

%\printanswers

\begin{document}
\begin{center}
  \sffamily\textbf{EXAMEN III}
\end{center}
\begin{flushright}
21 de noviembre de 2023
\end{flushright}

\begin{questions}
\begin{EnvFullwidth}
  \sffamily\textbf{Sección I.- Completar los siguientes enunciados con las
  palabras situadas en la parte inferior de la sección. Valor: 15 puntos
  (3 punto c/u).}
\end{EnvFullwidth}

% ex: ts=2 sw=2 sts=2 et filetype=tex
% SPDX-License-Identifier: CC-BY-SA-4.0

\question Se constituye por un identificador, un tipo de dato, un valor y un
          ámbito:
          \fillin[Variable]

% ex: ts=2 sw=2 sts=2 et filetype=tex
% SPDX-License-Identifier: CC-BY-SA-4.0

\question  

% ex: ts=2 sw=2 sts=2 et filetype=tex
% SPDX-License-Identifier: CC-BY-SA-4.0

\question Función que divide una cadena en una lista especificando un
          separador:
          \fillin[split()]

% ex: ts=2 sw=2 sts=2 et filetype=tex
% SPDX-License-Identifier: CC-BY-SA-4.0

\question \tf[V] El modo "r" abre un archivo de solo lectura y envía mensaje
          de error si el archivo no existe.

% ex: ts=2 sw=2 sts=2 et filetype=tex
% SPDX-License-Identifier: CC-BY-SA-4.0

\question \textbf{join()}: Función que retorna una cadena formada por los
          elementos de una lista separados por un delimitador.


% ex: ts=2 sw=2 sts=2 et filetype=tex
% SPDX-License-Identifier: CC-BY-SA-4.0

\begin{multicols}{4}
  \begin{enumerate}
    \item Variable
    \item Precedencia
    \item Lógico / Booleano
    \item eval()
    \item Pseudocódigo
    \item Intérprete
    \item Ordenado
    \item Identificador
    \item type()
    \item Sangría
    \item Operador
    \item Definido
    \item Tipo de operador
    \item Orientado a objetos
    \item Lenguaje de programación
  \end{enumerate}
\end{multicols}

\end{questions}

\begin{questions}
\begin{EnvFullwidth}
  \sffamily\textbf{Sección II.- Responder las siguientes afirmaciones con
  Falso (F) o Verdadero (V) según corres\-ponda. Valor 15 puntos (3 puntos c/u)}
\end{EnvFullwidth}

% ex: ts=2 sw=2 sts=2 et filetype=tex
% SPDX-License-Identifier: CC-BY-SA-4.0

\question \textbf{join()}: Función que retorna una cadena formada por los
          elementos de una lista separados por un delimitador.

% ex: ts=2 sw=2 sts=2 et filetype=tex
% SPDX-License-Identifier: CC-BY-SA-4.0

\question Función que divide una cadena en una lista especificando un
          separador:
          \fillin[split()]

% ex: ts=2 sw=2 sts=2 et filetype=tex
% SPDX-License-Identifier: CC-BY-SA-4.0

\question Se constituye por un identificador, un tipo de dato, un valor y un
          ámbito:
          \fillin[Variable]

% ex: ts=2 sw=2 sts=2 et filetype=tex
% SPDX-License-Identifier: CC-BY-SA-4.0

\question \tf[V] El modo "r" abre un archivo de solo lectura y envía mensaje
          de error si el archivo no existe.

% ex: ts=2 sw=2 sts=2 et filetype=tex
% SPDX-License-Identifier: CC-BY-SA-4.0

\question  


\end{questions}

%\newpage

\begin{questions}
\begin{EnvFullwidth}
  \sffamily\textbf{Sección III.- Escribir las instrucciones de manejo de
  archivos en lenguaje Python para cada uno de los casos descritos. Valor 20
  puntos (10 puntos c/u).
  }
\end{EnvFullwidth}

% ex: ts=2 sw=2 sts=2 et filetype=tex
% SPDX-License-Identifier: CC-BY-SA-4.0

\question Función que divide una cadena en una lista especificando un
          separador:
          \fillin[split()]

% ex: ts=2 sw=2 sts=2 et filetype=tex
% SPDX-License-Identifier: CC-BY-SA-4.0

\question \textbf{join()}: Función que retorna una cadena formada por los
          elementos de una lista separados por un delimitador.

%% ex: ts=2 sw=2 sts=2 et filetype=tex
% SPDX-License-Identifier: CC-BY-SA-4.0

\question  


\end{questions}

\begin{questions}
\begin{EnvFullwidth}
  \sffamily\textbf{Sección IV.- Definir las siguientes funciones de
  manipulación de cadenas equivalentes a las de Python, utilizando el propio
  lenguaje. Valor 20 puntos (10 puntos c/u).
  }
\end{EnvFullwidth}

% ex: ts=2 sw=2 sts=2 et filetype=tex
% SPDX-License-Identifier: CC-BY-SA-4.0

\question Función que divide una cadena en una lista especificando un
          separador:
          \fillin[split()]

% ex: ts=2 sw=2 sts=2 et filetype=tex
% SPDX-License-Identifier: CC-BY-SA-4.0

\question \textbf{join()}: Función que retorna una cadena formada por los
          elementos de una lista separados por un delimitador.

%% ex: ts=2 sw=2 sts=2 et filetype=tex
% SPDX-License-Identifier: CC-BY-SA-4.0

\question  


\end{questions}

\begin{questions}
\begin{EnvFullwidth}
  \sffamily\textbf{Sección V.- Resolver los siguientes problemas utilizando
  instrucciones del lenguaje Python. Valor 30 puntos.
  }
\end{EnvFullwidth}

% ex: ts=2 sw=2 sts=2 et filetype=tex
% SPDX-License-Identifier: CC-BY-SA-4.0

\question Función que divide una cadena en una lista especificando un
          separador:
          \fillin[split()]

%% ex: ts=2 sw=2 sts=2 et filetype=tex
% SPDX-License-Identifier: CC-BY-SA-4.0

\question \textbf{join()}: Función que retorna una cadena formada por los
          elementos de una lista separados por un delimitador.

%% ex: ts=2 sw=2 sts=2 et filetype=tex
% SPDX-License-Identifier: CC-BY-SA-4.0

\question  


\end{questions}

\end{document}
