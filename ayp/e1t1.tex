% ex: ts=2 sw=2 sts=2 et filetype=tex
% SPDX-License-Identifier: CC-BY-SA-4.0

\documentclass[10pt,addpoints]{exam}

\usepackage[utf8]{inputenc}
\usepackage[T1]{fontenc}
\usepackage[letterpaper]{geometry}
\usepackage{graphicx}
\usepackage{xcolor}
%\usepackage[spanish]{babel}

\definecolor{itesoprofundo}{RGB}{000, 066, 112}
\definecolor{itesomedio}{RGB}{112, 144, 183}
\definecolor{itesoclaro}{RGB}{151, 227, 255}

\pagestyle{headandfoot}
%\headrule
\lhead{\includegraphics[scale=0.5]{img/logo.png}}
\rhead{\color{itesoprofundo}\textbf{ALGORITMOS Y PROGRAMACIÓN}\\Departamento
       de Electrónica, Sistemas e Informática\\}
\footer{}{Página \thepage\ de \numpages}{}

\pointpoints{punto}{puntos}
\renewcommand{\solutiontitle}{\textbf{Solución: }}

%\printanswers

\begin{document}
\begin{center}
  \sffamily\textbf{EXAMEN I}
\end{center}
\begin{flushright}
12 de septiembre de 2023
\end{flushright}
Nombre:~\makebox[4in]{\hrulefill}
Puntos máximos: 100

\begin{questions}

\begin{EnvFullwidth}
  \sffamily\textbf{Sección I.- Completar los siguientes enunciados con las
  palabras situadas en la parte inferior de la sección. Valor: 10 puntos
  (1 punto c/u).}
\end{EnvFullwidth}

% ex: ts=2 sw=2 sts=2 et filetype=tex
% SPDX-License-Identifier: CC-BY-SA-4.0

\question Función que divide una cadena en una lista especificando un
          separador:
          \fillin[split()]

% ex: ts=2 sw=2 sts=2 et filetype=tex
% SPDX-License-Identifier: CC-BY-SA-4.0

\question \textbf{join()}: Función que retorna una cadena formada por los
          elementos de una lista separados por un delimitador.

% ex: ts=2 sw=2 sts=2 et filetype=tex
% SPDX-License-Identifier: CC-BY-SA-4.0

\question  

% ex: ts=2 sw=2 sts=2 et filetype=tex
% SPDX-License-Identifier: CC-BY-SA-4.0

\question Se constituye por un identificador, un tipo de dato, un valor y un
          ámbito:
          \fillin[Variable]

% ex: ts=2 sw=2 sts=2 et filetype=tex
% SPDX-License-Identifier: CC-BY-SA-4.0

\question \tf[V] El modo "r" abre un archivo de solo lectura y envía mensaje
          de error si el archivo no existe.

% ex: ts=2 sw=2 sts=2 et filetype=tex
% SPDX-License-Identifier: CC-BY-SA-4.0

\question Instrucción que evalua una condición si la condición anterior
          fue falsa: \fillin[elif]

% ex: ts=2 sw=2 sts=2 et filetype=tex
% SPDX-License-Identifier: CC-BY-SA-4.0

\question Instrucción que podemos detener la iteración actual del
          ciclo y continuar con la siguiente: \fillin[continue]

% ex: ts=2 sw=2 sts=2 et filetype=tex
% SPDX-License-Identifier: CC-BY-SA-4.0

\question Función que devuelve una secuencia de números:
          \fillin[range()]

% ex: ts=2 sw=2 sts=2 et filetype=tex
% SPDX-License-Identifier: CC-BY-SA-4.0

\question Prioridad de los operadores para evaluar una expresión:
          \fillin[Precedencia]

% ex: ts=2 sw=2 sts=2 et filetype=tex
% SPDX-License-Identifier: CC-BY-SA-4.0

\question Manejo de espacios al principio de la linea para delimitar
          bloques de código:
          \fillin[Sangría]


\end{questions}

\end{document}
