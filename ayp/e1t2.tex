% ex: ts=2 sw=2 sts=2 et filetype=tex
% SPDX-License-Identifier: CC-BY-SA-4.0

\documentclass[10pt,addpoints]{exam}

\usepackage[utf8]{inputenc}
\usepackage[T1]{fontenc}
\usepackage[letterpaper]{geometry}
\usepackage{graphicx}
\usepackage[table]{xcolor}
%\usepackage[spanish]{babel}
\usepackage{multicol}

\usepackage{tikz}
\newcommand{\pythagwidth}{3cm}
\newcommand{\pythagheight}{2cm}

\definecolor{itesoprofundo}{RGB}{000, 066, 112}
\definecolor{itesomedio}{RGB}{112, 144, 183}
\definecolor{itesoclaro}{RGB}{151, 227, 255}

\pagestyle{headandfoot}
%\headrule
\lhead{\includegraphics[scale=0.5]{img/logo.png}}
\rhead{\color{itesoprofundo}\textbf{ALGORITMOS Y PROGRAMACIÓN}\\Departamento
       de Electrónica, Sistemas e Informática\\}
\lfoot{Nombre:~\makebox[4.5in]{\hrulefill} }
\cfoot{}
\rfoot{Página \thepage\ de \numpages}

\pointpoints{punto}{puntos}
\renewcommand{\solutiontitle}{\textbf{Solución: }}
\renewcommand{\theenumi}{\alph{enumi})}

%\printanswers

\begin{document}
\begin{center}
  \sffamily\textbf{EXAMEN I}
\end{center}
\begin{flushright}
13 de febrero de 2024
\end{flushright}

\begin{questions}
\begin{EnvFullwidth}
  \sffamily\textbf{Sección I.- Completar los siguientes enunciados con las
  palabras situadas en la parte inferior de la sección. Valor: 30 puntos
  (3 punto c/u).}
\end{EnvFullwidth}

% ex: ts=2 sw=2 sts=2 et filetype=tex
% SPDX-License-Identifier: CC-BY-SA-4.0

\question Función que devuelve una secuencia de números:
          \fillin[range()]

% ex: ts=2 sw=2 sts=2 et filetype=tex
% SPDX-License-Identifier: CC-BY-SA-4.0

\question Instrucción que evalua una condición si la condición anterior
          fue falsa: \fillin[elif]

% ex: ts=2 sw=2 sts=2 et filetype=tex
% SPDX-License-Identifier: CC-BY-SA-4.0

\question Manejo de espacios al principio de la linea para delimitar
          bloques de código:
          \fillin[Sangría]

% ex: ts=2 sw=2 sts=2 et filetype=tex
% SPDX-License-Identifier: CC-BY-SA-4.0

\question Se constituye por un identificador, un tipo de dato, un valor y un
          ámbito:
          \fillin[Variable]

% ex: ts=2 sw=2 sts=2 et filetype=tex
% SPDX-License-Identifier: CC-BY-SA-4.0

\question \textbf{join()}: Función que retorna una cadena formada por los
          elementos de una lista separados por un delimitador.

% ex: ts=2 sw=2 sts=2 et filetype=tex
% SPDX-License-Identifier: CC-BY-SA-4.0

\question Instrucción que podemos detener la iteración actual del
          ciclo y continuar con la siguiente: \fillin[continue]

% ex: ts=2 sw=2 sts=2 et filetype=tex
% SPDX-License-Identifier: CC-BY-SA-4.0

\question  

% ex: ts=2 sw=2 sts=2 et filetype=tex
% SPDX-License-Identifier: CC-BY-SA-4.0

\question Función que divide una cadena en una lista especificando un
          separador:
          \fillin[split()]

% ex: ts=2 sw=2 sts=2 et filetype=tex
% SPDX-License-Identifier: CC-BY-SA-4.0

\question \tf[V] El modo "r" abre un archivo de solo lectura y envía mensaje
          de error si el archivo no existe.

% ex: ts=2 sw=2 sts=2 et filetype=tex
% SPDX-License-Identifier: CC-BY-SA-4.0

\question Prioridad de los operadores para evaluar una expresión:
          \fillin[Precedencia]


% ex: ts=2 sw=2 sts=2 et filetype=tex
% SPDX-License-Identifier: CC-BY-SA-4.0

\begin{multicols}{4}
  \begin{enumerate}
    \item Variable
    \item Precedencia
    \item Lógico / Booleano
    \item eval()
    \item Pseudocódigo
    \item Intérprete
    \item Ordenado
    \item Identificador
    \item type()
    \item Sangría
    \item Operador
    \item Definido
    \item Tipo de operador
    \item Orientado a objetos
    \item Lenguaje de programación
  \end{enumerate}
\end{multicols}

\end{questions}

\begin{questions}
\begin{EnvFullwidth}
  \sffamily\textbf{Sección II.- Evalúe las siguientes expresiones y
  determine su resultado. Incluya el desarrollo. Valor 10 puntos (5 puntos
  c/u)}
\end{EnvFullwidth}

% ex: ts=2 sw=2 sts=2 et filetype=tex
% SPDX-License-Identifier: CC-BY-SA-4.0

\begin{EnvFullwidth}
  Sean: $a=3$, $b=2$, $c=-1$, $d=1$
\end{EnvFullwidth}

% ex: ts=2 sw=2 sts=2 et filetype=tex
% SPDX-License-Identifier: CC-BY-SA-4.0

\question Es cada uno de los elementos que componen la población:

  \begin{oneparchoices}
    \CorrectChoice Individuo
    \choice Dato
    \choice Entrevista
  \end{oneparchoices}

% ex: ts=2 sw=2 sts=2 et filetype=tex
% SPDX-License-Identifier: CC-BY-SA-4.0

\question Conjunto de pasos, acciones o instrucciones necesarios para lograr 
un resultado  o resolver  un problema.

  \begin{oneparchoices}
    \CorrectChoice Algoritmo
    \choice Problema
    \choice Solución
    \choice Reseta de cocina
  \end{oneparchoices}


\end{questions}

\newpage

\begin{questions}
\begin{EnvFullwidth}
  \sffamily\textbf{Sección III.- Escribir un algoritmo en pseudocódigo y su
  representación en diagrama de flujo por cada uno de los siguientes
  problemas. Valor 20 puntos donde 10 puntos pseudocódigo y 10 puntos
  diagrama de flujo}
  %(20 puntos c/u, ).
\end{EnvFullwidth}

% ex: ts=2 sw=2 sts=2 et filetype=tex
% SPDX-License-Identifier: CC-BY-SA-4.0

\question Función que divide una cadena en una lista especificando un
          separador:
          \fillin[split()]

%% ex: ts=2 sw=2 sts=2 et filetype=tex
% SPDX-License-Identifier: CC-BY-SA-4.0

\question \textbf{join()}: Función que retorna una cadena formada por los
          elementos de una lista separados por un delimitador.

%% ex: ts=2 sw=2 sts=2 et filetype=tex
% SPDX-License-Identifier: CC-BY-SA-4.0

\question  


\end{questions}

\begin{questions}
\begin{EnvFullwidth}
  \sffamily\textbf{Sección IV.- Resolver los siguientes problemas utilizando
  instrucciones del lenguaje Python. Valor 40 puntos (20 puntos c/u).}
\end{EnvFullwidth}

% ex: ts=2 sw=2 sts=2 et filetype=tex
% SPDX-License-Identifier: CC-BY-SA-4.0

\question \textbf{join()}: Función que retorna una cadena formada por los
          elementos de una lista separados por un delimitador.

% ex: ts=2 sw=2 sts=2 et filetype=tex
% SPDX-License-Identifier: CC-BY-SA-4.0

\question Función que divide una cadena en una lista especificando un
          separador:
          \fillin[split()]

%% ex: ts=2 sw=2 sts=2 et filetype=tex
% SPDX-License-Identifier: CC-BY-SA-4.0

\question  


\end{questions}

\end{document}
