% ex: ts=2 sw=2 sts=2 et filetype=tex
% SPDX-License-Identifier: CC-BY-SA-4.0

\documentclass[12pt,addpoints]{exam}

\usepackage[utf8]{inputenc}
\usepackage[T1]{fontenc}
\usepackage[spanish]{babel}
\usepackage[letterpaper]{geometry}
\usepackage{pgfplots}

\pagestyle{headandfoot}
\headrule
\header{Cálculo Diferencial}{Examen}{CBTIS 246}
\footer{}{Página \thepage\ de \numpages}{}

\pointpoints{punto}{puntos}
\renewcommand{\solutiontitle}{\textbf{Solución: }}

\printanswers

\pgfplotsset{compat=1.18}

\begin{document}
\begin{center}
\fbox{\fbox{\parbox{5.5in}{\centering
Lee con atención cada pregunta y responde en
el espacio ubicado en la parte izquierda.
}}}
\end{center}

\vspace{5mm}

Nombre:\enspace\hrulefill

\vspace{5mm}

Grupo:\enspace\hrulefill
\enspace{}Grado:\enspace\hrulefill
\enspace{}Fecha:\enspace\hrulefill

\begin{questions}

\question bla bla

\begin{tikzpicture}[domain=0:4]
  \draw[dotted] (0,0) grid (3,2);
  \draw[->] (0,0) -- (1,0);
  
  %\draw[very thin,color=gray] (-0.1,-1.1) grid (3.9,3.9);
  %\draw[->] (-0.2,0) -- (4.2,0) node[right] x;
  %\draw[->] (0,-1.2) -- (0,4.2) node[above] {$f(x)$};
\end{tikzpicture}

\end{questions}

\end{document}
