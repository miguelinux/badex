% ex: ts=2 sw=2 sts=2 et filetype=tex
% SPDX-License-Identifier: CC-BY-SA-4.0

\documentclass[12pt,addpoints]{exam}

\usepackage[utf8]{inputenc}
\usepackage[T1]{fontenc}
%\usepackage[spanish]{babel}
\usepackage[letterpaper]{geometry}
\usepackage{pgfplots}
\usepackage{xcolor}

\pagestyle{headandfoot}
\headrule
\header{Cálculo Diferencial}{Examen}{CBTIS 246}
\footer{}{Página \thepage\ de \numpages}{}

\pointpoints{punto}{puntos}
\renewcommand{\solutiontitle}{\textbf{Solución: }}

\printanswers

\pgfplotsset{compat=1.18}

\begin{document}

%\begin{center}
%\fbox{\fbox{\parbox{5.5in}{\centering
%Lee con atención cada pregunta y responde en
%el espacio ubicado en la parte izquierda.
%}}}
%\end{center}

\vspace{5mm}

Nombre:\enspace\hrulefill

\vspace{5mm}

Grupo:\enspace\hrulefill
\enspace{}Grado:\enspace\hrulefill
\enspace{}Fecha:\enspace\hrulefill

\begin{questions}

\question Utiliza la tabla en el segundo y tercer renglón y escribe los
  resultados de la función $f(x) = sen(x)$ y $f(x) = cos(x)$ respectivamente.

  \textbf{Nota:} Recuerda de poner tu calculadora en \textbf{radianes}.

  \begin{tabular}{|l|c|c|c|c|c|c|c|c|c|c|c|c|c|}
    \hline & 0 & $\frac{\pi}{6}$  & $\frac{\pi}{3}$  & $\frac{\pi}{2}$ & $\frac{2\pi}{3}$%
               & $\frac{5\pi}{6}$ & $ \pi $  & $\frac{7\pi}{6}$ & $\frac{4\pi}{3}$%
               & $\frac{3\pi}{2}$  & $\frac{5\pi}{3}$ & $\frac{11\pi}{6}$ & $2\pi$ \\
    \hline $sen(x)$ & \hspace{7mm} & \hspace{7mm} & \hspace{7mm}
           & \hspace{7mm} & \hspace{7mm} & \hspace{7mm} & \hspace{7mm}
           & \hspace{7mm} & \hspace{7mm} & \hspace{7mm} & \hspace{7mm}
           & \hspace{7mm} & \hspace{7mm} \\
    \hline $cos(x)$ & \hspace{7mm} & \hspace{7mm} & \hspace{7mm}
           & \hspace{7mm} & \hspace{7mm} & \hspace{7mm} & \hspace{7mm}
           & \hspace{7mm} & \hspace{7mm} & \hspace{7mm} & \hspace{7mm}
           & \hspace{7mm} & \hspace{7mm} \\ \hline
  \end{tabular}

\begin{tikzpicture}[domain=0:13]
  \draw[dotted] (-0.3,-4) grid (13,4);

  \draw[->] (0,0) --  (13,0) node[right] {$x$};
  \draw[->] (0,-4) -- (0,4) node[above] {$f(x)$};

  \path (0.2,-0.2) node[] {0};
  \path (2.2,-0.2) node[] {1};
  \path (4.2,-0.2) node[] {2};
  \path (6.2,-0.2) node[] {3};
  \path (8.2,-0.2) node[] {4};
  \path (10.2,-0.2) node[] {5};
  \path (12.2,-0.2) node[] {6};

  \path (-0.7,-4) node[] {-1};
  \path (-0.7,-3) node[] {-0.75};
  \path (-0.7,-2) node[] {-0.50};
  \path (-0.7,-1) node[] {-0.25};
  \path (-0.7,1) node[] {0.25};
  \path (-0.7,2) node[] {0.50};
  \path (-0.7,3) node[] {0.75};
  \path (-0.7,4) node[] {1};

  %\draw[color=blue] plot[id=sin] function{sin(x/2)*4}  node[right] {$f(x) = \sin x$};
  %\draw[color=red] plot[id=cos]  function{cos(x/2)*4}  node[right] {$f(x) = \cos x$};
  
\end{tikzpicture}

\clearpage

\question Utiliza la tabla en el segundo renglón y escribe los
  resultados de la función $f(x) = tan(x)$.

  \textbf{Nota:} Recuerda de poner tu calculadora en \textbf{radianes}.

  \begin{tabular}{|l|c|c|c|c|c|c|c|c|c|c|c|c|c|}
    \hline & 0 & $\frac{\pi}{6}$  & $\frac{\pi}{3}$  & $\frac{\pi}{2}$ & $\frac{2\pi}{3}$%
               & $\frac{5\pi}{6}$ & $ \pi $  & $\frac{7\pi}{6}$ & $\frac{4\pi}{3}$%
               & $\frac{3\pi}{2}$ \\
    \hline $tan(x)$ 
           & \hspace{7mm} & \hspace{7mm} & \hspace{7mm} & \hspace{7mm}
           & \hspace{7mm} & \hspace{7mm} & \hspace{7mm} & \hspace{7mm}
           & \hspace{7mm} & \hspace{7mm} \\ \hline
  \end{tabular}

\begin{tikzpicture}
  \draw[dotted] (-0.3,-7) grid (5,7);

  \draw[->] (0,0) --  (6,0) node[right] {$x$};
  \draw[->] (0,-7) -- (0,7) node[above] {$f(x)$};

  \path (0.2,-0.2) node[] {0};
  \path (1.2,-0.2) node[] {1};
  \path (2.2,-0.2) node[] {2};
  \path (3.2,-0.2) node[] {3};
  \path (4.2,-0.2) node[] {4};
  \path (5.2,-0.2) node[] {5};

  \path (-0.7,-7) node {-7};
  \path (-0.7,-6) node {-6};
  \path (-0.7,-5) node {-5};
  \path (-0.7,-4) node {-4};
  \path (-0.7,-3) node {-3};
  \path (-0.7,-2) node {-2};
  \path (-0.7,-1) node {-1};
  \path (-0.7,1) node {1};
  \path (-0.7,2) node {2};
  \path (-0.7,3) node {3};
  \path (-0.7,4) node {4};
  \path (-0.7,5) node {5};
  \path (-0.7,6) node {6};
  \path (-0.7,7) node {7};

  %\draw[color=red] plot[id=tan1, domain=0:1.43] function{tan(x)};
  %\draw[color=red] plot[id=tan2, domain=1.71:4.57] function{tan(x)} node[right] {$f(x) = \tan x$};
  
\end{tikzpicture}

\end{questions}

\end{document}
